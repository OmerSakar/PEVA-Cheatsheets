\section{Lecture 8 Closed queueing networks}
\textbf{Gordon Newell QNs:} always stable.

\hrule
\settowidth{\MyLen}{\texttt{Service Demand.}}
\begin{tabular}{@{}p{\the\MyLen}@{}p{\linewidth-\the\MyLen}@{}}
\verb!Visit count!	&  $V_j = \sum^M_{i=1} V_ir_{i,j}$, Solavble if one of the Vs is given\\
\verb!Service Demand!	&  $D_i = V_i \cdot E[S_i]$. Highest service demand is bottleneck\\
\end{tabular}
\hrule
If we increase K, then $X(K) \rightarrow \frac{1}{D_b}$ and $\rho_b \rightarrow 1$. For non-bottleneck stations, $\rho_i \rightarrow \frac{D_i}{D_b} < 1$

\textbf{Gordon And Newell product-form:} $I(M,K)$ is every possible combination of $\underline{n} = (n_1, n_2 ... n_M)$. $P(\underline{N} = \underline{n}) = \frac{1}{G(M,K)} \prod^M_{i=1} D_i^{n_i}$ with normalisation constant $G(M,K) = \sum_{n \in I(M,K)} \prod_{i=1}^M D_i^{n_i}$. Size of $I(M,K) = {M+K-1 \choose M-1}$

\textbf{MVA:} 

\hrule
\settowidth{\MyLen}{\texttt{2nd factorial moment.}}
\begin{tabular}{@{}p{\the\MyLen}@{}p{\linewidth-\the\MyLen}@{}}
$E[\hat{R}_i(k)]$ 	&  $(E[N_i(k-1)] + 1)D_i$\\
$E[\hat{R}(k)]$ 	&  $\sum E[\hat{R}_i(K)]$\\
$X(K)$ 				&  $\frac{K}{E[\hat{R}(k)]}$\\
$E[N_i(k)]$ 		&  $X(k)E[\hat{R}_i(k)]$\\
\verb!For IS node!  &  $E[R_i](K) = E[S_i]$ and $E[\hat{R}_i](K) = D_i$
\end{tabular}
\newline
\hrule
\includegraphics[scale=0.2]{images/MVA.png}

\textbf{Asumptotic bounds:}
\hrule
\settowidth{\MyLen}{\texttt{Saturation point.}}
\begin{tabular}{@{}p{\the\MyLen}@{}p{\linewidth-\the\MyLen}@{}}
\verb!X(K) upperbound! 	&  $X(K) \leq \min\{\frac{K}{E[Z] + D_\Sigma}, \frac{1}{D_+}\}$\\
\verb!E[R] lower bound! 	&  $E[\hat{R}(K)] \geq \max\{E[Z] + D_\Sigma, KD_+\}$\\
\verb!! 	&  \\
\verb!Saturation point! 	&  $K^*  = \frac{D_\Sigma + E[Z]}{D_+}$. The integer part of $K^*$ can be interpreted as the maximum number of customers that could be accommodated without any queueing when all the service times are of deterministic length. Stated differently, if the number of customers is larger than K* we are sure that queueing effects in the network contribute to the response times.\\
\end{tabular}
\hrule

\textbf{Bard-Schweiter Approximation:}
MVA with $E[\hat{R}_i(K)] \approx (\frac{K-1}{K} E[N_i(K)] + 1) D_i$ and as a first guess $E[N_i](k) = \frac{K}{M}$.

\textbf{Balanced Queueing Networks:} Assume all station ahve the same service demands and $E[N_i(K)] = \frac{K}{M}$.
\hrule
\settowidth{\MyLen}{\texttt{Saturation point.}}
\begin{tabular}{@{}p{\the\MyLen}@{}p{\linewidth-\the\MyLen}@{}}
$E[\hat{R}_i(K)]$ 	&  $\frac{D(K+M-1)}{M}$\\
$E[\hat{R}_(K)]$ 	&  $D(K+M-1)$\\
$X(K)$ 	&  $\frac{K}{D(K+M-1)}$\\
$\rho_i(K)$ 	&  $\frac{K}{K+M-1}$\\
\verb!Simple bounds! 	&  \\
			& $\frac{K}{D_+(K+M-1)} \leq X(K) \leq \frac{K}{D_-(K+M-1)}$\\
			& $D_-(K+M-1) \leq E[\hat{R}(K)] \leq D_+(K+M-1)$\\
\verb!Tighter bounds! 	&  \\
\verb!Performance! 	&  best if $D = \bar{D} = \frac{D_\Sigma}{•M}$\\
\verb!Throughput! 	&  $X(K) \leq \min\{\frac{K}{\bar{D}(K+M-1)}, 
\frac{1}{D_+}\}$\\
&  $\max\{\bar{D}(K+M-1), KD_+\} \leq E[\hat{R}(K)] \leq D_+(K-1) + D_\Sigma$\\\\
\end{tabular}
\hrule